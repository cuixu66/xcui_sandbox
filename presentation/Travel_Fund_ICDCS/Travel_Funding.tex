\documentclass[letterpaper]{article}
\usepackage[left=1in, right=1in, top=1in, bottom=1in]{geometry}% sets the margins
\usepackage{times}% sets the fonts.
\usepackage{graphicx}% for importing figures using \includegraphics{} command
\usepackage[section]{placeins}% ``tries'' to keep floating objects (figure, table) within section boundaries
\usepackage{xspace}

% some frequently used abberviations1
\newcommand{\ie}{\emph{i.e.,}}
\newcommand{\eg}{\emph{e.g.,}}
\newcommand{\sect}[1]{Section~\ref{#1}}
\newcommand{\fig}[1]{Fig.~\ref{#1}}
\newcommand{\ycsb}{{YCSB}\xspace}
\newcommand{\lpaper}{{MicroFuge}\xspace}
\newcommand{\memcached}{{Memcached}\xspace}

\begin{document}

\title{MicroFuge: A Middleware Approach to Providing Performance Isolation in
  Cloud Storage Systems}% document title

\author{Akshay K. Singh, Xu Cui, Benjamin Cassell, Bernard Wong and Khuzaima Daudjee\\
  \{ak5singh, xcui, becassel, bernard, kdaudjee\}@uwaterloo.ca}

\date{}
\maketitle% ``builds'' the title

% a label acts as a key and can be used to refer
% to that section later. e.g., \ref{sec:intro}
% reference numbers will be generated automatically

\begin{abstract}

Most cloud providers improve resource utilization by having multiple tenants
share the same resources. However, this comes at the cost of reduced isolation
between tenants, which can lead to inconsistent and unpredictable performance.
This performance variability is a significant impediment for tenants running
services with strict latency deadlines.  Providing predictable performance is
particularly important for cloud storage systems. The storage system is the
performance bottleneck for many cloud-based services and therefore often
determines their overall performance characteristics.

In this paper, we introduce \lpaper, a new distributed caching and scheduling
middleware that provides performance isolation for cloud storage systems.
\lpaper addresses the performance isolation problem by building an
empirically-driven performance model of the underlying storage system based on
measured data.  Using this model, \lpaper provides adaptive deadline-aware
cache eviction, scheduling and load-balancing policies, minimizing deadline
misses.  \lpaper can also perform early rejection of requests that are unlikely
to make their deadlines. Using workloads from the \ycsb benchmark on an EC2
deployment, we show that adding \lpaper to the storage stack substantially
reduces the deadline miss rate of a distributed storage system. Additionally,
\lpaper is significantly more effective at reducing deadline misses than
\memcached, a popular distributed caching middleware.

\end{abstract}

\end{document}
